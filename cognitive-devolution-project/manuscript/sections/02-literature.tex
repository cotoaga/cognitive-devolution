\section{Literature Review: Fragmented Recognition and Systematic Blindness}

\subsection{The Peripheral Scouts}

A careful survey of contemporary scholarship reveals a curious phenomenon: researchers at the edges of multiple disciplines have independently begun recognizing the energetic dimensions of cognition, yet these insights remain unintegrated, failing to coalesce into a unified framework that could challenge the dominant paradigm of costless information processing.

In management science, \citet{bratianu2020} claims to use ``for the first time a thermodynamics approach'' to understand knowledge dynamics, proposing knowledge entropy as an organizing principle for organizational cognition. That such a claim could be made in 2020 (decades after information theory established entropy measures) reveals the profound isolation between knowledge management and physical sciences. \citet{bratianu2020b} extend this framework, arguing that knowledge manifests in three forms (rational, emotional, spiritual) that transform through ``energy-like processes,'' yet they stop short of recognizing that these are not metaphorical but literal energy transformations.

In neuroscience, researchers have begun quantifying the metabolic costs of cognition with increasing precision. \citet{jamadar2025} demonstrates that goal-directed cognition requires only 5\% more energy than resting brain activity: a finding that paradoxically reveals both the brain's efficiency and the critical importance of that marginal energy investment. \citet{wiehler2022} provide mechanistic evidence that cognitive control exertion leads to glutamate accumulation in the lateral prefrontal cortex, establishing a direct biochemical basis for mental fatigue. These findings suggest that ``cognitive work'' is not merely analogous to physical labor but operates through similar energetic constraints.

In physics and information theory, \citet{stonier1996} proposed treating information as a basic property of the universe alongside matter and energy, arguing for fundamental interconvertibility between information and energy. Yet this theoretical breakthrough remains largely unknown to knowledge management scholars, who continue treating information as an abstract, costless commodity.

\subsection{The Mainstream Blindness}

Despite these peripheral insights, the dominant discourse in knowledge management, organizational theory, and educational policy proceeds as if cognition were energetically neutral. The vast literature on the ``knowledge economy'' \citep{powell2004}, ``learning organizations'' \citep{senge1990}, and ``competency-based education'' \citep{mulder2007} treats knowledge as an infinitely reproducible resource constrained only by access and transmission bandwidth.

Consider the influential SECI model (Socialization, Externalization, Combination, Internalization) of knowledge creation by \citet{nonaka1995}, cited over 30,000 times. While brilliantly mapping knowledge transformation modes, it contains no recognition that each transformation requires energy investment, that maintaining tacit knowledge demands continuous metabolic support, or that externalization represents an entropic process that degrades multidimensional understanding into linear documentation.

Similarly, the burgeoning literature on artificial intelligence and knowledge work (from \citet{brynjolfsson2014} ``Second Machine Age'' to \citet{susskind2020} ``Future of the Professions'') focuses on computational capability and pattern recognition while ignoring the energetic basis that distinguishes biological from silicon cognition. These works treat the replacement of human expertise as a matter of algorithmic sophistication rather than recognizing it as the logical endpoint of a century-long process of cognitive energy disinvestment.

\subsection{Cognitive Capitalism's Energy Blindness}

The critical literature on ``cognitive capitalism'' \citep{moulierboutang2007, vercellone2007} comes closest to recognizing the exploitation of mental resources yet still fails to ground this in thermodynamic reality. Moulier-Boutang distinguishes between ``labor-power'' (physical energy expenditure) and ``invention-power'' (cognitive functions) without recognizing that invention-power also requires literal energy investment: not metaphorical ``mental energy'' but actual glucose metabolism, ATP consumption, and entropic heat dissipation.

This blindness extends to the platform economy literature. \citet{zuboff2019} ``surveillance capitalism'' brilliantly exposes behavioral data extraction but doesn't recognize that platforms are essentially entropy accelerators, harvesting the organized complexity of human cognition while investing nothing in its maintenance or development. \citet{srnicek2017} ``platform capitalism'' identifies data as the new oil but misses that, unlike oil, cognitive resources require continuous energy investment to prevent degradation.

\subsection{The Expertise Literature Gap}

The extensive literature on expertise development (from \citet{ericsson2006} deliberate practice to \citet{kahneman2009} conditions for expert intuition) meticulously documents the time requirements for skill acquisition (the famous ``10,000 hours''), but rarely acknowledges these as energy investment requirements. When researchers note that expertise requires ``effort'' or ``cognitive load,'' they treat these as psychological rather than thermodynamic phenomena.

Even sophisticated critiques of expert systems, from \citet{dreyfus1979} to \citet{collins2010}, focus on the irreducibility of tacit knowledge without recognizing that this irreducibility stems from its high-energy state. Tacit knowledge resists formalization not because it is mysteriously ineffable but because maintaining it requires continuous metabolic investment that cannot be captured in static representations.

\subsection{The Integration Imperative}

What emerges from this review is not an absence of relevant insights but their tragic fragmentation. Neuroscientists measure metabolic costs without connecting to knowledge theory. Management scholars invoke entropy without thermodynamic grounding. Physicists theorize information-energy equivalence without application to human cognition. Critical theorists expose cognitive exploitation without energetic foundation.

This fragmentation is not accidental but structural: a consequence of disciplinary boundaries that mirror the very vectorization this paper critiques. Just as education has collapsed multidimensional cognition into specialized competencies, academia has partitioned the study of knowledge into non-communicating silos, preventing recognition of the unified thermodynamic reality underlying all cognitive phenomena.

The task before us is not to discover new facts but to synthesize existing insights into a framework that reveals what disciplinary fragmentation has hidden: the systematic transformation of high-energy spherical cognition into low-energy vectors suitable for algorithmic consumption, and the thermodynamic impossibility of maintaining cognitive sovereignty without corresponding energy investment.
