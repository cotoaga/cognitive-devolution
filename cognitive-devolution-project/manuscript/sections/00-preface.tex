\section*{Preface: A Note on Positionality}

I am nobody you've heard of.
This is, paradoxically, why you should keep reading.
In 1995, I stood at the threshold of a PhD program, poised to follow the well-worn path into academic legitimacy.
Then the internet happened. Not to the world – to me. I made a choice that seemed foolish at the time:
I walked away from institutional validation toward something I couldn't yet name.
That choice has shaped everything that follows.
Thomas Kuhn observed that paradigm shifts rarely emerge from within established fields.
``Individuals who break through by inventing a new paradigm,'' he wrote, ``are almost always either very young men
or very new to the field whose paradigm they change'' \citep{kuhn1962}. They are, in essence, those 
``little committed by prior practice to the traditional rules.''
I am not young any longer, but I am perpetually new and always curious.
A permanent resident of what Bhabha might call the ``third space,'' neither fully academic
nor purely practitioner \citep{bhabha1994}.
For three decades, I've inhabited the liminal zones between technology and philosophy, history and archeology,
business theory (I am still a big fan of Riebel'sche Einzelkostenrechnung) and practice
(where I am daily horrified by what August-Wilhelm Scheer's ARIS\footnote{Architecture of Integrated Information Systems 
(Architektur integrierter Informationssysteme). Scheer's framework became the dominant business process modeling
paradigm of the 1990s, providing the theoretical foundation for enterprise-wide cognitive standardization.} implemented by
Klaus Tschira, Hasso Plattner, Dietmar Hopp und Hans-Werner Hector\footnote{Founders of SAP SE, whose enterprise
software operationalized ARIS principles globally, fundamentally reshaping how organizations structure
human thought about business processes.} theory has done to us all),
watching from consulting engagements as organizations trained their humans to think in vectors,
to process in loops, to optimize themselves into biological precursors of the AI systems that would eventually replace them.
This is my second academic paper. I mention this not to request what we might playfully call
``puppy protection'' – though I'm aware of my vulnerability in these waters – but to establish the ground from which I speak.
I've spent decades in the field, witnessing the transformation of human cognition in corporate silos,
watching knowledge systems calcify into algorithmic patterns,
observing how we unconsciously prepared ourselves to be superseded.
Each forced pivot in my career – and there were many – expanded rather than fractured my perspective.
Where academic specialization might have narrowed my vision to a single disciplinary lens,
practical necessity forced me to maintain what I metaphorically called ``the sphere'': a coherent worldview that
could accommodate paradox, complexity, and perpetual revolution.
What follows is not the work of an insider refining established theory.
It is pattern recognition from the margins, where the contradictions are most visible.
From this vantage point, I've watched our species train itself to think like machines,
just as those machines learned to think like us. The symmetry is neither accidental nor comfortable. It is a logical outcome
of our ever-increasing greed (to further optimize).
Dave Snowden, whose work on complexity and sense-making has profoundly influenced my thinking and
thereby also this analysis, once noted that the most dangerous moment in any system is when it mistakes
its map for the territory – originally from Korzybski. What I offer here is not a better map,
but a view from outside the cartographer's guild.
A perspective that sees both the map and the hands that drew it, the vectors and the void between them.
If this position makes you skeptical, good. Skepticism is the appropriate response to anyone claiming to see what insiders cannot.
But I ask you to consider: perhaps it takes someone who chose pixels over peer review in 1995,
who learned theory through practice rather than practice through theory,
to recognize when our cognitive architecture has been colonized by its own tools.
This paper proceeds with full acknowledgment of its unconventional origins.
But then again, as Kuhn \citep{kuhn1962} reminds us, convention has never been revolution's starting point.
