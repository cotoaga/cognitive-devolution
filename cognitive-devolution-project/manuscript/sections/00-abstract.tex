\begin{abstract}
\noindent This paper presents a thermodynamic framework for understanding the systematic 
transformation of human cognition from multidimensional ``spherical'' architectures to 
unidimensional ``vectors'' optimized for algorithmic processing. 
Drawing on evidence from historical educational systems, contemporary organizational failures, 
and the architecture of artificial intelligence, 
we demonstrate that knowledge represents a high-energy state requiring continuous
investment to maintain organizational complexity. Our analysis reveals that a century-long process
 aimed at optimizing knowledge management has accidentally standardized and modularized human cognition
 – documented in what we term ``confession literature'' from education, management, and technology sectors –
 has created the precise conditions for AI replacement of human expertise.
 We trace this transformation from ancient Greek academies
 (requiring 20+ years of energy investment) through medieval guilds, the Bologna Process,
 to contemporary micro-credentials approaching thermodynamic zero.
 The paper introduces the equation:
 $\text{Cognitive Sovereignty} = \frac{\text{Energy Invested}}{\text{Time}} \times \text{Resistance to Extraction}$,
 where resistance correlates with architectural complexity.
 Evidence from continuously increasing transformation failure rates (70-88\%), AI implementation disasters,
 and German engineering education resistance validates our thermodynamic model.
 We conclude that the choice facing individuals and institutions is binary:
 invest energy in sphere reconstruction or accept entropic dissolution.
 Physics doesn't negotiate.
\end{abstract}
