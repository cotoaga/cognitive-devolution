\begin{abstract}
Why do the vast majority of organizational transformations fail? Why can AI suddenly replace knowledge workers who spent decades developing expertise? The answer lies in a century-long process we never intended: training humans to think like machines.

This paper reveals how education and management systems have systematically transformed human cognition from multidimensional ``spheres'', capable of navigating complexity through infinite cross-domain connections, into one-dimensional ``vectors'' optimized for procedural efficiency. Using thermodynamic analysis, we demonstrate that knowledge, like any organized system, requires continuous energy investment to maintain its complexity. Without this investment, expertise degrades to procedures, then to rituals, and finally to patterns simple enough for algorithms to replicate.

We trace this degradation from ancient Greek academies (20+ years developing interconnected wisdom) through medieval guilds and universities (7--14 years of embodied mastery) to today's micro-credentials (40 hours of standardized modules). The 1999 Bologna Process, which standardized European higher education, alone reduced cognitive investment by 70\% while fragmenting knowledge into tradeable credits (ECTS). Contemporary ``confession literature'', papers where educators and consultants inadvertently document their role in standardizing human thought, reveals that this wasn't conspiracy but optimization: each local decision to increase efficiency reduced the energy invested in developing human capability.

Our thermodynamic equation---Cognitive Sovereignty = (Energy Invested/Time) $\times$ Resistance to Extraction---explains why leading German technical universities (TU9) are desperately trying to revive the Diplom-Ingenieur they sacrificed to Bologna, while 6-week bootcamps gleefully produce GPT-5.x fodder. The equation predicts and explains the very high rate of AI project abandonment: organizations discover that vectors can optimize but only spheres can adapt.

The implications are stark. We haven't reached ``peak human'' and been surpassed by superior AI. We've depleted human cognitive development to the point where simple pattern-matching systems can replicate our degraded outputs. The choice facing individuals and institutions is thermodynamically binary: invest the decades required to develop genuine expertise or accept replacement by systems that process our documented patterns more efficiently than we do.

Physics doesn't negotiate. Neither should we. The thermodynamic framework we present isn't complex---it's simply the physics of why your skills get rusty and your organization falls apart without maintenance, applied to human knowledge.
\end{abstract}